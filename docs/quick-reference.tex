%!TEX program=xelatex
% This cheatsheet is based on the template
% provided at https://gist.github.com/alexander-yakushev/c773543bf9a957749f79.
\documentclass[10pt,english,landscape]{article}
    \usepackage{multicol}
    \usepackage{calc}
    \usepackage[landscape]{geometry}
    \usepackage{color,graphicx,overpic}
    
    % https://tex.stackexchange.com/a/192067
    \usepackage{fontawesome}
      \newfontfamily{\FA}{[FontAwesome.otf]}
    
    % Known issue in {fontspec}: https://github.com/wspr/fontspec/issues/312#issuecomment-342125206
    % Since {fontspec} is included only when compiling with XeTeX, we guard
    % our fix accordingly.
    \usepackage{ifxetex}
        \ifxetex
            \let\latinencoding\relax
            \usepackage{fontspec}
            \setmainfont{Segoe UI}
            \setmonofont{Consolas}
        \fi
    
    % \usepackage[T1]{fontenc}
    % \usepackage[bitstream-charter]{mathdesign}
    \usepackage[utf8]{inputenc}
    \usepackage{url}
    \usepackage{amsfonts}
    \usepackage{array,booktabs}
    \usepackage{textcomp}
    \usepackage[usenames,dvipsnames,table]{xcolor}
    \usepackage[most]{tcolorbox}
    \usepackage{menukeys}
    \usepackage{tabularx}
    \usepackage{multirow}
    \usepackage{colortbl}
    \usepackage{tikz}
    \usepackage{environ}
    \usepackage{braket}
    
    \usetikzlibrary{calc}
    \pgfdeclarelayer{background}
    \pgfdeclarelayer{foreground}
    \pgfsetlayers{background,main,foreground}
    
    \geometry{top=-0.5cm,left=1cm,right=1cm,bottom=1cm}
    
    \pagestyle{empty} % Turn off header and footer
    
    % \renewcommand\rmdefault{phv} % Arial
    % \renewcommand\sfdefault{phv} % Arial
    
    % Redefine section commands to use less space
    \makeatletter
    \renewcommand{\section}{\@startsection{section}{1}{0mm}%
      {-1ex plus -.5ex minus -.2ex}%
      {0.5ex plus .2ex}%x
      {\normalfont\large\bfseries}}
    \renewcommand{\subsection}{\@startsection{subsection}{2}{0mm}%
      {-1explus -.5ex minus -.2ex}%
      {0.5ex plus .2ex}%
      {\normalfont\normalsize\bfseries}}
    \renewcommand{\subsubsection}{\@startsection{subsubsection}{3}{0mm}%
      {-1ex plus -.5ex minus -.2ex}%
      {1ex plus .2ex}%
      {\normalfont\small\bfseries}}
    \makeatother
    
    \setcounter{secnumdepth}{0} % Don't print section numbers
    \setlength{\parindent}{0pt}
    \setlength{\parskip}{0pt plus 0.5ex}
    
    \definecolor{TableHead}{rgb}{0.353, 0.329, 0.667}
    \definecolor{TableRow}{rgb}{0.816, 0.812, 0.902}
    
    \NewEnviron{keysref}[1]{
      % \begin{center}
      \smallskip
      \begin{tikzpicture}
          \rowcolors{1}{}{TableRow}
    
          \node (tbl) [inner sep=0pt] {
            \begin{tabular}{p{2.5cm}p{5.05cm}}
              \rowcolor{TableHead}
              \multicolumn{2}{l}{\normalsize\textbf{\color{white}{#1}}}\parbox{0pt}{\rule{0pt}{0.3ex+\baselineskip}}\\
              \BODY
              \arrayrulecolor{TableHead}\specialrule{.17em}{0em}{.2em}
            \end{tabular}};
          \begin{pgfonlayer}{background}
            \draw[rounded corners=2pt,top color=TableHead,bottom color=TableHead, draw=white]
            ($(tbl.north west)-(0,-0.05)$) rectangle ($(tbl.north east)-(0.0,0.15)$);
            \draw[rounded corners=2pt,top color=TableHead,bottom color=TableHead, draw=white]
            ($(tbl.south west)-(0.0,-0.11)$) rectangle ($(tbl.south east)-(-0.0,-0.02)$);
          \end{pgfonlayer}
        \end{tikzpicture}
      % \end{center}
    }
    
    % https://tex.stackexchange.com/a/102523
    \newcommand{\forceindent}[1]{\leavevmode{\parindent=#1\indent}}
    
    %% CUSTOM NOTATION %%
    
    \newcommand{\qs}{Q\#}
    \newcommand{\unixlike}{\hfill\faApple\faLinux}
    \newcommand{\ctrllike}{\hfill\faWindows\faLinux}
    
    %%%%%%%%%%%%%%%%%%%%%%%%%%%%%%%%%%%%%%%%%%%%%%%%%%%%%%%%%%%%%%%%%%%%%%%%%%%%%%
    
    \begin{document}
    
    \raggedright\
    
    % \begin{center}
      \Large{\qs~Tutorial Quick Reference}
    % \end{center}
    
    \footnotesize
    \begin{multicols}{3}
    
      \section{Working with Tutorials}

      \begin{keysref}{Command Line Basics}
        Change directory    & \texttt{cd \emph{dirname}} \\
        Go to home          & \texttt{cd \textasciitilde} \\
        Go up one directory & \texttt{cd ..} \\
        Make new directory  & \texttt{mkdir \emph{dirname}} \\
        Open current \newline directory in VS Code & \texttt{code .} \\
      \end{keysref}
    
      \begin{keysref}{Building Project and Running Tests}
        Change directory to \newline project directory & \texttt{cd \emph{project-dir}} \\
        Build solution                    & \texttt{dotnet build} \\
        Run all unit tests                & \texttt{dotnet test} \\
      \end{keysref}
    
      \begin{keysref}{VS Code Keyboard Shortcuts}
        Open command \newline palette   & \keys{Ctrl + Shift + P} \ctrllike \newline
                                          \keys{\cmd + \shift + P} \hfill\faApple \\
        Show terminal                   & \keys{Ctrl + \`{}} \ctrllike \newline
                                          \keys{\cmd + \`{}} \hfill\faApple \\
        Open folder                     & \keys{Ctrl + K} then \keys{Ctrl + O} \ctrllike \newline
                                          \keys{\cmd + K} then \keys{\cmd + O} \hfill\faApple \\
      \end{keysref}
    
      \begin{keysref}{Documentation}
        Quantum \newline Development Kit & \url{https://docs.microsoft.com/quantum} \\
        \qs~ Language \newline Reference & \url{https://docs.microsoft.com/en-us/quantum/quantum-qr-intro} \\
        \qs~ Library \newline Reference  & \url{https://docs.microsoft.com/en-us/qsharp/api/} \\
      \end{keysref}
      
      \section{\qs~Language}
    
      \begin{keysref}{Primitive Types}
        64-bit integers         & \texttt{Int}    \\
        Double-precision floats & \texttt{Double} \\
        Booleans                & \texttt{Bool}   \newline 
                                  \texttt{true} or \texttt{false} \\
        Qubits                  & \texttt{Qubit}  \\
        Pauli operators         & \texttt{Pauli}  \newline
                                  \texttt{PauliI}, \texttt{PauliX}, \texttt{PauliY}, or \texttt{PauliZ} \\
        Measurement \newline results     & \texttt{Result} \newline
                                  \texttt{Zero} or \texttt{One} \\
        Sequences of \newline integers   & \texttt{Range}  \newline
                                  e.g.: 1..10 or 5..-1..0 \\
        Strings                 & \texttt{String} \\
      \end{keysref}
    
      \columnbreak%\
    
      \section{\qs~Language (continued)}
    
      \begin{keysref}{Derived Types}
        Arrays                  & \texttt{\emph{elementType}[]} \\
        Tuples                  & \texttt{(\emph{type0}, \emph{type1}, ...)} \\
        Functions               & \texttt{\emph{inputType} -> \emph{outputType}} \newline
                                  e.g.: \texttt{ArcTan2 : (Double, Double) -> Double} \\
        Operations              & \texttt{\emph{inputType} => \emph{outputType} : \emph{variants}} \newline
                                  e.g.: \texttt{H : (Qubit => () : Adjoint, Controlled)} \\
      \end{keysref}
    
      \begin{keysref}{Function and Operation Definitions}
        Functions & \texttt{function \emph{Name}(\emph{in0} : \emph{type0}, ...) : \emph{returnType} \{} \newline
                    \texttt{\hphantom{....}// \emph{function body}} \newline
                    \texttt{\}} \\
        Operations & \texttt{operation \emph{Name}(\emph{in0} : \emph{type0}, ...) : \emph{returnType} \{} \newline
                     \texttt{\hphantom{....}body \{ ... \}} \newline
                     \texttt{\hphantom{....}adjoint \{ ... \}} \newline
                     \texttt{\hphantom{....}controlled \{ ... \}} \newline
                     \texttt{\hphantom{....}adjoint controlled \{ ... \}} \newline
                     \texttt{\}} \\
      \end{keysref}

      \begin{keysref}{Calling w/ Functors}
        Adjoint call & \texttt{(Adjoint \emph{Name})(\emph{operationInputs})} \\
        Controlled \newline call & \texttt{(Controlled \emph{Name})(\emph{controlQubits}, \emph{operationInputs})} \\
      \end{keysref}
    
      \begin{keysref}{Variables}
        Declare immutable & \texttt{let \emph{name} = \emph{value}} \\
        Declare mutable & \texttt{mutable \emph{name} = \emph{value}} \\
        Update mutable & \texttt{set \emph{name} = \emph{value}} \\
      \end{keysref}
    
      \begin{keysref}{Arrays}
        Allocation           & \texttt{mutable \emph{name} = new \emph{Type}[\emph{length}]} \\
        Length               & \texttt{Length(\emph{name})} \\
        $i$th element \newline (0-based indexing) & \texttt{\emph{name}[i]} \\
        Array literal        & \texttt{[\emph{value0}; \emph{value1}; ...]} \newline
                               e.g.: \texttt{[true; false; true]} \\
        Slicing (subarray)   & \texttt{let \emph{name} = \emph{name}[\emph{start}..\emph{end}]} \\
      \end{keysref}
    
      \begin{keysref}{Control Flow}
        For loop                   & \texttt{for (\emph{ind} in \emph{range}) \{ ... \}} \newline 
                                     e.g.: \texttt{for (i in 0..N-1) \{ ... \}} \\
        Repeat-until-success loop  & \texttt{repeat \{ ... \} \newline until \emph{condition} \newline fixup \{ ... \}} \\
        Conditional \newline statement      & \texttt{if \emph{cond1} \{ ... \} \newline elif \emph{cond2} \{ ... \} \newline else \{ ... \}}\\
      \end{keysref}
    
      \columnbreak%\
      \section{Qubits and Operations on Qubits}
    
      \begin{keysref}{Qubits}
        Allocation           & \texttt{using (\emph{name} = \emph{Qubit}[\emph{length}] \{} \newline
                               \texttt{\hphantom{....}//} Qubits in \texttt{\emph{name}} start in $\ket{0}$. \newline
                               \texttt{\hphantom{....}...} \newline
                               \texttt{\hphantom{....}//} Qubits must be returned to $\ket{0}$. \newline
                               \texttt{\}} \\
      \end{keysref}
    
      \begin{keysref}{Standard Library Functions and Operations}
        Pauli operations          & \texttt{I : (Qubit => ())} \newline
                                    \texttt{X : (Qubit => ())} \newline
                                    \texttt{Y : (Qubit => ())} \newline
                                    \texttt{Z : (Qubit => ())} \\
        Hadamard                  & \texttt{H : (Qubit => ())} \\
        Controlled-NOT            & \texttt{CNOT : ((control : Qubit, \newline target : Qubit) => ())} \\
        Measure one qubit in Pauli $Z$ basis & \texttt{M : Qubit => Result} \\
        Perform joint measurement of qubits in given Pauli bases & \texttt{Measure : (Pauli[], Qubit[]) => Result} \\
        Rotate about given Pauli axis   & \texttt{R : (Pauli, Double, Qubit) => ()} \\
        Rotate about Pauli $X$, $Y$, $Z$ axes & \texttt{Rx : (Double, Qubit) => ()} \newline
                                          \texttt{Ry : (Double, Qubit) => ()} \newline
                                          \texttt{Rz : (Double, Qubit) => ()} \\
        Reset qubit to $\ket{0}$     & \texttt{Reset : Qubit => ()} \\
        Reset qubits to $\ket{0..0}$ & \texttt{ResetAll : Qubit[] => ()} \\
        Apply a gate to each qubit   & \texttt{ApplyToEach : ((Qubit => ()), Qubit[]) => ()} \\
        Display a log message & \texttt{Message : String -> ()} \\
      \end{keysref}
    
    
    \end{multicols}
    
\end{document} 